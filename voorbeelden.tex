\section{Voorbeelden}

\subsection{Transitietabel NFA}
\label{ex:nfatbl}

\begin{nfa}
  \node[initial,state]   (S)                    {$S$};
  \node[state]           (A) [below right of=S] {$A$};
  \node[state,accepting] (E) [above right of=A] {$E$};
  
  \path (S) edge []           node         {$a$}          (E)
            edge []           node         {$a,\epsilon$} (A)
        (A) edge [loop below] node         {$b$}          (A)
            edge [bend right] node         {$a$}          (E)
        (E) edge [loop right] node         {$a,b$}        (E)
            edge [bend right] node [above] {$\epsilon$}   (A);
  \addvmargin{1mm}
\end{nfa}

De bijhorende transitietabel voor de gegeven NFA:

\begin{center}
\begin{tabular}{r|r|r}
  $Q$ & $\Sigma_\epsilon$ & $\powerset(Q)$ \\ \hline
  $S$ & $a$ & $\{A,E\}$ \\
  $S$ & $\epsilon$ & $\{A\}$ \\
  $A$ & $a$ & $\{E\}$ \\
  $A$ & $b$ & $\{A\}$ \\
  $E$ & $a$ & $\{E\}$ \\
  $E$ & $b$ & $\{E\}$ \\
  $E$ & $\epsilon$ & $\{A\}$ \\ \hline
  $S$ & $b$ & $\emptyset$ \\
  $A$ & $\epsilon$ & $\emptyset$ \\
\end{tabular}
\end{center}

\subsection{NFA naar RE}
\label{ex:gnfa}

\begin{nfa}
  \node[initial,state]   (S)                    {$S$};
  \node[state]           (A) [below right of=S] {$A$};
  \node[state,accepting] (E) [above right of=A] {$E$};
  
  \path (S) edge []           node         {$a$}          (E)
            edge []           node         {$a,\epsilon$} (A)
        (A) edge [loop below] node         {$b$}          (A)
            edge [bend right] node         {$a$}          (E)
        (E) edge [loop right] node         {$a,b$}        (E)
            edge [bend right] node [above] {$\epsilon$}   (A);
  \addvmargin{1mm}
\end{nfa}

\paragraph{Stap 1} Generaliseren

\begin{nfa}
  \node[initial,state]   (SN)                    {$q_s$};
  \node[state]           (S)  [right of=SN]      {$S$};
  \node[state]           (A)  [below right of=S] {$A$};
  \node[state]           (E)  [above right of=A] {$E$};
  \node[state,accepting] (EN) [right of=E]       {$q_e$};
  
  \path (SN) edge []           node         {$\epsilon$}   (S)
        (S)  edge []           node         {$a$}          (E)
             edge []           node         {$a|\epsilon$} (A)
        (A)  edge [loop below] node         {$b$}          (A)
             edge [bend right] node         {$a$}          (E)
        (E)  edge [loop above] node         {$a|b$}        (E)
             edge [bend right] node [above] {$\epsilon$}   (A)
             edge []           node         {$\epsilon$}   (EN);
  \addvmargin{1mm}
\end{nfa}

\paragraph{Stap 2} Reductie (met vereenvoudiging)

Reductie 1:\\
\begin{nfa}
  \node[initial,state]   (SN)                    {$q_s$};
  \node[state]           (S)  [right of=SN]      {$S$};
  \node[state]           (A)  [below right of=S] {$A$};
  \node[state]           (E)  [above right of=A] {$E$};
  \node[state,accepting] (EN) [right of=E]       {$q_e$};
  
  \path (SN) edge []           node {$\epsilon$}   (S)
        (S)  edge []           node {$a$}          (E)
             edge []           node {$a|\epsilon$} (A)
        (A)  edge [loop below] node {$b$}          (A)
             edge []           node {$a$}          (E)
        (E)  edge [loop above] node {$a|b|b^*a$}   (E)
             edge []           node {$\epsilon$}   (EN);
  \addvmargin{1mm}
\end{nfa}

\begin{nfa}
  \node[initial,state]   (SN)                  {$q_s$};
  \node[state]           (S)  [right of=SN]    {$S$};
  \node[state]           (E)  [right=2cm of S] {$E$};
  \node[state,accepting] (EN) [right of=E]     {$q_e$};
  
  \path (SN) edge []           node {$\epsilon$}         (S)
        (S)  edge [bend right] node {$(a|\epsilon)b^*a$} (E)
             edge [bend left]  node {$a$}                (E)
        (E)  edge [loop above] node {$b|b^*a$}           (E)
             edge []           node {$\epsilon$}         (EN);
  \addvmargin{1mm}
\end{nfa}

\begin{nfa}
  \node[initial,state]   (SN)                  {$q_s$};
  \node[state]           (S)  [right of=SN]    {$S$};
  \node[state]           (E)  [right=3cm of S] {$E$};
  \node[state,accepting] (EN) [right of=E]     {$q_e$};
  
  \path (SN) edge []           node {$\epsilon$}         (S)
        (S)  edge []           node {$(a|\epsilon)b^*a$} (E)
        (E)  edge [loop above] node {$b|b^*a$}              (E)
             edge []           node {$\epsilon$}         (EN);
  \addvmargin{1mm}
\end{nfa}

Reductie 2:\\
\begin{nfa}
  \node[initial,state]   (SN)                  {$q_s$};
  \node[state]           (S)  [right of=SN]    {$S$};
  \node[state,accepting] (EN) [right=4cm of S] {$q_e$};
  
  \path (SN) edge [] node {$\epsilon$}                   (S)
        (S)  edge [] node {$(a|\epsilon)b^*a(b|b^*a)^*$} (EN);
  \addvmargin{1mm}
\end{nfa}

\begin{nfa}
  \node[initial,state]   (SN)                  {$q_s$};
  \node[state]           (S)  [right of=SN]    {$S$};
  \node[state,accepting] (EN) [right=4cm of S] {$q_e$};
  
  \path (SN) edge [] node {$\epsilon$}                   (S)
        (S)  edge [] node {$(a|\epsilon)b^*a(a|b)^*$} (EN);
  \addvmargin{1mm}
\end{nfa}

Reductie 3:\\
\begin{nfa}
  \node[initial,state]   (SN)                   {$q_s$};
  \node[state,accepting] (EN) [right=4cm of SN] {$q_e$};
  
  \path (SN) edge [] node {$(a|\epsilon)b^*a(a|b)^*$} (EN);
  \addvmargin{1mm}
\end{nfa}

\begin{nfa}
  \node[initial,state]   (SN)                   {$q_s$};
  \node[state,accepting] (EN) [right=4cm of SN] {$q_e$};
  
  \path (SN) edge [] node {$b^*a(a|b)^*$} (EN);
  \addvmargin{1mm}
\end{nfa}

\paragraph{Stap 3} De RE voor de NFA is ``$b^*a(a|b)^*$".

\subsection{NFA naar DFA}
\label{ex:nfadfa}

\subsubsection{Voorbeeld 1}

\begin{nfa}
  \node[initial,state]   (S)                    {$S$};
  \node[state]           (A) [below right of=S] {$A$};
  \node[state,accepting] (E) [above right of=A] {$E$};
  
  \path (S) edge []           node         {$a$}          (E)
            edge []           node         {$a,\epsilon$} (A)
        (A) edge [loop below] node         {$b$}          (A)
            edge [bend right] node         {$a$}          (E)
        (E) edge [loop right] node         {$a,b$}        (E)
            edge [bend right] node [above] {$\epsilon$}   (A);
  \addvmargin{1mm}
\end{nfa}

We construeren een tabel met alle toestanden van $Q_d = \powerset(Q_n)$ en de epsilon-bereikbare functie:

\begin{center}
\begin{tabular}{r|r}
  $q \in Q_d$ & $eb(q)$ \\ \hline
  $\{\}$ & $\{\}$ \\
  $\{S\}$ & $\{S,A\}$ \\
  $\{A\}$ & $\{A\}$ \\
  $\{E\}$ & $\{A,E\}$ \\
  $\{S,A\}$ & $\{S,A\}$ \\
  $\{S,E\}$ & $\{S,A,E\}$ \\
  $\{A,E\}$ & $\{A,E\}$ \\
  $\{S,A,E\}$ & $\{S,A,E\}$ \\
\end{tabular}
\end{center}

We stellen nu de verzamelingen voor $\delta_n(Q_d,a)$ en $\delta_n(Q_d,b)$ op:

\begin{center}
\begin{tabular}{r|r|r|r}
  $q \in Q_d$ & $eb(q)$ & $\delta_n(q,a)$ & $\delta_n(q,b)$ \\ \hline
  $\{\}$ & $\{\}$ & $\{\}$ & $\{\}$ \\
  $\{S\}$ & $\{S,A\}$ & $\{A,E\}$ & $\{\}$ \\
  $\{A\}$ & $\{A\}$ & $\{E\}$ & $\{A\}$ \\
  $\{E\}$ & $\{A,E\}$ & $\{E\}$ & $\{E\}$ \\
  $\{S,A\}$ & $\{S,A\}$ & $\{A,E\}$ & $\{A\}$ \\
  $\{S,E\}$ & $\{S,A,E\}$ & $\{A,E\}$ & $\{E\}$ \\
  $\{A,E\}$ & $\{A,E\}$ & $\{E\}$ & $\{A,E\}$ \\
  $\{S,A,E\}$ & $\{S,A,E\}$ & $\{A,E\}$ & $\{A,E\}$ \\
\end{tabular}
\end{center}

Tenslotte stellen nu de verzamelingen voor $\delta_d(Q_d,a)=eb(\delta_n(Q_d,a))$ en $\delta_d(Q_d,b)=eb(\delta_n(Q_d,b))$ op:

\begin{center}
\begin{tabular}{r|r|r|r|r|r}
  $q \in Q_d$ & $eb(q)$ & $\delta_n(q,a)$ & $\delta_n(q,b)$ & $\delta_d(q,a)$ & $\delta_d(q,b)$ \\ \hline
  $\{\}$ & $\{\}$ & $\{\}$ & $\{\}$ & $\{\}$ & $\{\}$ \\
  $\{S\}$ & $\{S,A\}$ & $\{A,E\}$ & $\{\}$ & $\{A,E\}$ & $\{\}$ \\
  $\{A\}$ & $\{A\}$ & $\{E\}$ & $\{A\}$ & $\{A,E\}$ & $\{A\}$ \\
  $\{E\}$ & $\{A,E\}$ & $\{E\}$ & $\{E\}$ & $\{A,E\}$ & $\{A,E\}$ \\
  $\{S,A\}$ & $\{S,A\}$ & $\{A,E\}$ & $\{A\}$ & $\{A,E\}$ & $\{A\}$ \\
  $\{S,E\}$ & $\{S,A,E\}$ & $\{A,E\}$ & $\{E\}$ & $\{A,E\}$ & $\{A,E\}$ \\
  $\{A,E\}$ & $\{A,E\}$ & $\{E\}$ & $\{A,E\}$ & $\{A,E\}$ & $\{A,E\}$ \\
  $\{S,A,E\}$ & $\{S,A,E\}$ & $\{A,E\}$ & $\{A,E\}$ & $\{A,E\}$ & $\{A,E\}$ \\
\end{tabular}
\end{center}

De resulterende DFA:

\begin{nfa}
  \node[initial,state]   (S)                    {$\{S,A\}$};
  \node[state]           (A) [below right of=S] {$\{A\}$};
  \node[state,accepting] (E) [above right of=A] {$\{A,E\}$};
  
  \path (S) edge []           node {$a$}   (E)
            edge []           node {$b$}   (A)
        (A) edge [loop below] node {$b$}   (A)
            edge []           node {$a$}   (E)
        (E) edge [loop right] node {$a,b$} (E);
  \addvmargin{1mm}
\end{nfa}

\subsubsection{Voorbeeld 2}

\begin{nfa}
  \node[initial,state]   (q1)               {$q_1$};
  \node[state,accepting] (q2) [right of=q1] {$q_2$};
  
  \path (q1) edge [loop above] node {$a,b$} (q1)
             edge []           node {$b$}   (q2);
  \addvmargin{1mm}
\end{nfa}

We construeren een tabel met alle toestanden van $Q_d = \powerset(Q_n)$ en de epsilon-bereikbare functie:

\begin{center}
\begin{tabular}{r|r}
  $q \in Q_d$ & $eb(q)$ \\ \hline
  $\{\}$ & $\{\}$ \\
  $\{q_1\}$ & $\{q_1\}$ \\
  $\{q_2\}$ & $\{q_2\}$ \\
  $\{q_1,q_2\}$ & $\{q_1,q_2\}$ \\
\end{tabular}
\end{center}

We stellen nu de verzamelingen voor $\delta_n(Q_d,a)$ en $\delta_n(Q_d,b)$ op:

\begin{center}
\begin{tabular}{r|r|r|r}
  $q \in Q_d$ & $eb(q)$ & $\delta_n(q,a)$ & $\delta_n(q,b)$ \\ \hline
  $\{\}$ & $\{\}$ & $\{\}$ & $\{\}$ \\
  $\{q_1\}$ & $\{q_1\}$ & $\{q_1\}$ & $\{q_1,q_2\}$ \\
  $\{q_2\}$ & $\{q_2\}$ & $\{\}$ & $\{\}$ \\
  $\{q_1,q_2\}$ & $\{q_1,q_2\}$ & $\{q_1\}$ & $\{q_1,q_2\}$ \\
\end{tabular}
\end{center}

Tenslotte stellen nu de verzamelingen voor $\delta_d(Q_d,a)=eb(\delta_n(Q_d,a))$ en $\delta_d(Q_d,b)=eb(\delta_n(Q_d,b))$ op:

\begin{center}
\begin{tabular}{r|r|r|r|r|r}
  $q \in Q_d$ & $eb(q)$ & $\delta_n(q,a)$ & $\delta_n(q,b)$ & $\delta_d(q,a)$ & $\delta_d(q,b)$ \\ \hline
  $\{\}$ & $\{\}$ & $\{\}$ & $\{\}$ & $\{\}$ & $\{\}$ \\
  $\{q_1\}$ & $\{q_1\}$ & $\{q_1\}$ & $\{q_1,q_2\}$ & $\{q_1\}$ & $\{q_1,q_2\}$ \\
  $\{q_2\}$ & $\{q_2\}$ & $\{\}$ & $\{\}$ & $\{\}$ & $\{\}$ \\
  $\{q_1,q_2\}$ & $\{q_1,q_2\}$ & $\{q_1\}$ & $\{q_1,q_2\}$ & $\{q_1\}$ & $\{q_1,q_2\}$ \\
\end{tabular}
\end{center}

De resulterende DFA:

\begin{nfa}
  \node[initial,state]   (q1)                     {$\{q_1\}$};
  \node[state,accepting] (q2) [right=1.5cm of q1] {$\{q_1,q_2\}$};
  
  \path (q1) edge [loop above] node         {$a$} (q1)
             edge [bend left]  node [above] {$b$} (q2)
        (q2) edge [loop above] node         {$b$} (q2)
             edge [bend left]  node [below] {$a$} (q1);
  \addvmargin{1mm}
\end{nfa}

\subsection{Parti\"ele DFA naar complete DFA}
\label{ex:dfatotal}

\textit{In dit voorbeeld laten we de bogen van $q \in F$ naar $q_t$ weg om de leesbaarheid te bevorderen. Het is echter belangrijk deze niet te vergeten om een echte complete DFA te vormen!}

\begin{nfa}
  \node[initial,state]   (q1)               {$q_1$};
  \node[state]           (q3) [right of=q1] {$q_3$};
  \node[state]           (q2) [above of=q3] {$q_2$};
  \node[state,accepting] (q4) [right of=q3] {$q_4$};
  \node[state]           (q5) [below of=q3] {$q_5$};
  
  \path (q1) edge [] node {$a$} (q2)
             edge [] node {$b$} (q3)
             edge [] node {$c$} (q5)
        (q2) edge [] node {$c$} (q4)
        (q3) edge [] node {$c$} (q4)
        (q5) edge [] node {$d$} (q4);
  \addvmargin{1mm}
\end{nfa}

Voor de gegeven DFA doen we de omzetting $\delta_{partieel} \longrightarrow \delta_{totaal}$. We voegen een ``trash'' toestand toe waar we alle ontbrekende bogen naar afleiden.

\begin{nfa}
  \node[initial,state]   (q1)               {$q_1$};
  \node[state]           (q3) [right of=q1] {$q_3$};
  \node[state]           (q2) [above of=q3] {$q_2$};
  \node[state,accepting] (q4) [right of=q3] {$q_4$};
  \node[state]           (q5) [below of=q3] {$q_5$};
  \node[state,dashed]    (qt) [left of=q1]  {$q_t$};
  
  \path (q1) edge []                  node              {$a$}       (q2)
             edge []                  node              {$b$}       (q3)
             edge []                  node              {$c$}       (q5)
             edge [dashed,bend right] node [above]      {$d$}       (qt)
        (q2) edge []                  node              {$c$}       (q4)
             edge [dashed,bend right] node [above left] {$a,b,d$}   (qt)
        (q3) edge []                  node              {$c$}       (q4)
             edge [dashed,bend left]  node              {$a,b,d$}   (qt)
        (q5) edge []                  node              {$d$}       (q4)
             edge [dashed,bend left]  node              {$a,b,c$}   (qt)
        (qt) edge [dashed,loop left]  node              {$a,b,c,d$} (qt);
  \addvmargin{1mm}
\end{nfa}

De resulterende automaat is nog steeds een geldige DFA, maar heeft nu de toestand $q_t$ van waaruit het niet mogelijk is om de eindtoestand $q_4$ te bereiken.

\subsection{DFA f-gelijk}
\label{ex:dfafeq}

We willen de f-gelijke toestanden van een DFA bepalen waarvan we eerst de parti\"ele overgangsfunctie totaal hebben gemaakt. We nemen het resultaat van het voorbeeld in sectie \ref{ex:dfatotal}.

We beginnen met de initiatie van het algoritme om de f-gelijke toestanden te bepalen en stellen de initi\"ele graaf $V$.

\begin{ugraph}
  \node[state]        (q1)                     {$q_1$};
  \node[state]        (q2) [above right of=q1] {$q_2$};
  \node[state]        (q3) [right of=q2]       {$q_3$};
  \node[state]        (q4) [below right of=q3] {$q_4$};
  \node[state]        (q5) [below left of=q4]  {$q_5$};
  \node[state,dashed] (qt) [left of=q5]        {$q_t$};
  
  \path (q1) edge []       node {} (q4)
        (q2) edge []       node {} (q4)
        (q3) edge []       node {} (q4)
        (q5) edge []       node {} (q4)
        (qt) edge [dashed] node {} (q4);
  \addvmargin{1mm}
\end{ugraph}

We kiezen twee willekeurige knopen die nog ``onbeslist'' zijn, of m.a.w. die nog niet verbonden zijn in $V$. We nemen $q_1$ en $q_5$ en vinden dat $(\delta(q_1, d), \delta(q_5, d)) = (q_t, q_4) \in V$. We voegen de boog $(q_1, q_5)$ toe aan $V$.

\begin{ugraph}
  \node[state]        (q1)                     {$q_1$};
  \node[state]        (q2) [above right of=q1] {$q_2$};
  \node[state]        (q3) [right of=q2]       {$q_3$};
  \node[state]        (q4) [below right of=q3] {$q_4$};
  \node[state]        (q5) [below left of=q4]  {$q_5$};
  \node[state,dashed] (qt) [left of=q5]        {$q_t$};
  
  \path (q1) edge []       node {} (q4)
             edge []       node {} (q5)
        (q2) edge []       node {} (q4)
        (q3) edge []       node {} (q4)
        (q5) edge []       node {} (q4)
        (qt) edge [dashed] node {} (q4);
  \addvmargin{1mm}
\end{ugraph}

We kiezen opnieuw twee willekeurige knopen die nog ``onbeslist'' zijn, of m.a.w. die nog niet verbonden zijn in $V$. We nemen $q_1$ en $q_2$ en vinden dat $(\delta(q_1, c), \delta(q_5, c)) = (q_5, q_4) \in V$. We voegen de boog $(q_1, q_2)$ toe aan $V$.

\begin{ugraph}
  \node[state]        (q1)                     {$q_1$};
  \node[state]        (q2) [above right of=q1] {$q_2$};
  \node[state]        (q3) [right of=q2]       {$q_3$};
  \node[state]        (q4) [below right of=q3] {$q_4$};
  \node[state]        (q5) [below left of=q4]  {$q_5$};
  \node[state,dashed] (qt) [left of=q5]        {$q_t$};
  
  \path (q1) edge []       node {} (q2)
             edge []       node {} (q4)
             edge []       node {} (q5)
        (q2) edge []       node {} (q4)
        (q3) edge []       node {} (q4)
        (q5) edge []       node {} (q4)
        (qt) edge [dashed] node {} (q4);
  \addvmargin{1mm}
\end{ugraph}

We herhalen de iteratiestap tot we er geen enkel paar knopen meer voldoet aan de voorwaarden en we de volgende graaf $V$ bekomen:

\begin{ugraph}
  \node[state]        (q1)                     {$q_1$};
  \node[state]        (q2) [above right of=q1] {$q_2$};
  \node[state]        (q3) [right of=q2]       {$q_3$};
  \node[state]        (q4) [below right of=q3] {$q_4$};
  \node[state]        (q5) [below left of=q4]  {$q_5$};
  \node[state,dashed] (qt) [left of=q5]        {$q_t$};
  
  \path (q1) edge []       node {} (q2)
             edge []       node {} (q3)
             edge []       node {} (q4)
             edge []       node {} (q5)
        (q2) edge []       node {} (q4)
             edge []       node {} (q5)
        (q3) edge []       node {} (q4)
             edge []       node {} (q5)
        (q5) edge []       node {} (q4)
        (qt) edge [dashed] node {} (q1)
             edge [dashed] node {} (q2)
             edge [dashed] node {} (q3)
             edge [dashed] node {} (q4)
             edge [dashed] node {} (q5);
  \addvmargin{1mm}
\end{ugraph}

Vervolgens bepalen we de complementgraaf $G$ van $V$:

\begin{ugraph}
  \node[state]        (q1)                     {$q_1$};
  \node[state]        (q2) [above right of=q1] {$q_2$};
  \node[state]        (q3) [right of=q2]       {$q_3$};
  \node[state]        (q4) [below right of=q3] {$q_4$};
  \node[state]        (q5) [below left of=q4]  {$q_5$};
  \node[state,dashed] (qt) [left of=q5]        {$q_t$};
  
  \path (q2) edge [] node {} (q3);
  \addvmargin{1mm}
\end{ugraph}

De complementgraaf $G$ bevat een boog tussen $q_2$ en $q_3$, welke een component vormen en dus onderling f-gelijk zijn. Alle andere componenten bevatten slechts een enkele toestand.

\subsection{DFA f-gelijk (partities verfijnen)}
\label{ex:dfafeqpart}

We willen de f-gelijke toestanden van een DFA bepalen waarvan we eerst de parti\"ele overgangsfunctie totaal hebben gemaakt. We nemen het resultaat van het voorbeeld in sectie \ref{ex:dfatotal}.

De eerste partitie wordt opgesteld als een partitie die de toestanden in de eind- en niet-eindtoestanden verdeelt:

\begin{equation*}
P_0 = \{\{q_1,q_2,q_3,q_5,q_t\},\{q_4\}\}
\end{equation*}

Nu voeren we iteratiestappen uit. We partitioneren elke deelverzameling van de partities zodanig dat voor elk symbool, er vanuit beiden toestanden een overgang is naar dezelfde deelverzameling van de partitie.

\begin{equation*}
P_1 = \{\{q_1\},\{q_2,q_3\},\{q_5\},\{q_t\},\{q_4\}\}
\end{equation*}

Voor de gegeven DFA loopt het partitioneren af na een enkele stap. Het is niet mogelijk om de partitie nog verder te verfijnen. Elke deelverzameling van de partitie bevat nu de f-gelijke toestanden van de DFA.

\subsection{DFA minimaliseren}
\label{ex:dfamin}

We zullen een DFA minimaliseren, zoals die gegeven is aan het begin van sectie \ref{ex:dfatotal}. We moeten geen knopen verwijderen bij deze DFA. Het compleet maken van de DFA wordt uitgevoerd in sectie \ref{ex:dfatotal} en in sectie \ref{ex:dfafeq} (of \ref{ex:dfafeqpart}) worden de f-gelijke toestanden bepaalt. We stellen de volgende DFA op:

\begin{nfa}
  \node[initial,state]   (q1)                     {$q_1$};
  \node[state]           (q2) [above right of=q1] {$q_{23}$};
  \node[state,accepting] (q4) [below right of=q2] {$q_4$};
  \node[state]           (q5) [below right of=q1] {$q_5$};
  \node[state,dashed]    (qt) [left of=q1]        {$q_t$};
  
  \path (q1) edge []                  node              {$a,b$}     (q2)
             edge []                  node              {$c$}       (q5)
             edge [dashed,bend right] node [above]      {$d$}       (qt)
        (q2) edge []                  node              {$c$}       (q4)
             edge [dashed,bend right] node [above left] {$a,b,d$}   (qt)
        (q5) edge []                  node              {$d$}       (q4)
             edge [dashed,bend left]  node              {$a,b,c$}   (qt)
        (qt) edge [dashed,loop left]  node              {$a,b,c,d$} (qt);
  \addvmargin{1mm}
\end{nfa}

Tenslotte verwijderen we nog de ``trash''-toestand (of hulptoestand) die we eerder hebben toegevoegd om een totale overgangsfunctie te bekomen. De resulterende DFA is minimaal:

\begin{nfa}
  \node[initial,state]   (q1)                     {$q_1$};
  \node[state]           (q2) [above right of=q1] {$q_{23}$};
  \node[state,accepting] (q4) [below right of=q2] {$q_4$};
  \node[state]           (q5) [below right of=q1] {$q_5$};
  
  \path (q1) edge [] node {$a,b$} (q2)
             edge [] node {$c$}   (q5)
        (q2) edge [] node {$c$}   (q4)
        (q5) edge [] node {$d$}   (q4);
  \addvmargin{1mm}
\end{nfa}

\subsection{Taal met NFA kleiner dan minimale DFA}
\label{ex:nfaldfamin}

Een voorbeeld van een taal waarvoor een NFA bestaat die kleiner is dan DFA$_{min}$ is de taal over $\Sigma = \{a,b\}$ bepaald door een reguliere expressie gegeven als
\begin{equation*}
  (a|b)^*a(a|b)(a|b)
\end{equation*}

Een NFA die dezelfde taal bepaalt is:

\begin{nfa}
  \node[initial,state]   (S)              {$S$};
  \node[state]           (A) [right of=S] {$A$};
  \node[state]           (B) [right of=A] {$B$};
  \node[state,accepting] (E) [right of=B] {$E$};
  
  \path (S) edge [loop above] node {$a,b$} (S)
            edge []           node {$a$}   (A)
        (A) edge []           node {$a,b$} (B)
        (B) edge []           node {$a,b$} (E);
  \addvmargin{1mm}
\end{nfa}

We stellen een verkorte tabel op beginnende bij toestand $S$. Er is geen enkele boog met $\epsilon$, dus $\forall a \in \Sigma: \delta_n(q,a)=\delta_d(q,a)$. Elke toestand van $Q_d$ krijgt een alias om de leesbaarheid te bevorderen.

\begin{center}
\begin{tabular}{r|r|r|r}
  $q \in Q_d$ & $eb(q)$ & $\delta_n(q,a)=\delta_d(q,a)$ & $\delta_n(q,b)=\delta_d(q,b)$ \\ \hline
  $1=\{S\}$ & $\{S\}$ & $\{S,A\}$ & $\{S\}$\\
  $2=\{S,A\}$ & $\{S,A\}$ & $\{S,A,B\}$ & $\{S,B\}$\\
  $3=\{S,A,B\}$ & $\{S,A,B\}$ & $\{S,A,B,E\}$ & $\{S,B,E\}$\\
  $4=\{S,B\}$ & $\{S,B\}$ & $\{S,A,E\}$ & $\{S,E\}$\\
  $5=\{S,A,B,E\}$ & $\{S,A,B,E\}$ & $\{S,A,B,E\}$ & $\{S,B,E\}$\\
  $6=\{S,B,E\}$ & $\{S,B,E\}$ & $\{S,A,E\}$ & $\{S,E\}$\\
  $7=\{S,A,E\}$ & $\{S,A,E\}$ & $\{S,A,B\}$ & $\{S,B\}$\\
  $8=\{S,E\}$ & $\{S,E\}$ & $\{S,A\}$ & $\{S\}$\\
\end{tabular}
\end{center}

We kunnen vervolgende de bijhorende DFA construeren:

\begin{nfa}
  \node[initial,state]   (1)                    {$1$};
  \node[state]           (2) [above left of=1]  {$2$};
  \node[state]           (4) [above right of=2] {$4$};
  \node[state,accepting] (7) [right of=4]       {$7$};
  \node[state]           (3) [right of=7]       {$3$};
  \node[state,accepting] (5) [right of=3]       {$5$};
  \node[state,accepting] (6) [below right of=5] {$6$};
  \node[state,accepting] (8) [below left of=6]  {$8$};
  
  \path (1) edge [loop below]    node {$b$} (1)
            edge []              node {$a$} (2)
        (2) edge []              node {$b$} (4)
            edge [bend left=90]  node {$a$} (3)
        (4) edge [bend left=40]  node {$a$} (7)
            edge [bend right=10] node {$b$} (8)
        (7) edge []              node {$a$} (3)
            edge []              node {$b$} (4)
        (3) edge []              node {$a$} (5)
            edge [bend right=10] node {$b$} (6)
        (5) edge [loop above]    node {$a$} (5)
            edge []              node {$b$} (6)
        (6) edge [bend left=15]  node {$a$} (7)
            edge []              node {$b$} (8)
        (8) edge []              node {$a$} (2)
            edge []              node {$b$} (1);
  \addvmargin{1mm}
\end{nfa}

We proberen om de gevonden DFA nog verder te minimaliseren door de f-verschillende toestanden te bepalen met het standaard algoritme. We bekomen de volgende graaf:

\begin{ugraph}
  \node[state] (1)              {$1$};
  \node[state] (2) [above right of=1] {$2$};
  \node[state] (4) [right of=2] {$4$};
  \node[state] (7) [below right of=4] {$7$};
  \node[state] (3) [below of=7] {$3$};
  \node[state] (5) [below left of=3] {$5$};
  \node[state] (6) [left of=5]  {$6$};
  \node[state] (8) [above left of=6]  {$8$};
  
  % INIT
  \path (1) edge [] node {} (7)
        (2) edge [] node {} (7)
        (4) edge [] node {} (7)
        (3) edge [] node {} (7)
        (1) edge [] node {} (5)
        (2) edge [] node {} (5)
        (4) edge [] node {} (5)
        (3) edge [] node {} (5)
        (1) edge [] node {} (6)
        (2) edge [] node {} (6)
        (4) edge [] node {} (6)
        (3) edge [] node {} (6)
        (1) edge [] node {} (8)
        (2) edge [] node {} (8)
        (4) edge [] node {} (8)
        (3) edge [] node {} (8);
  % REPEAT
  \path (1) edge [] node {} (3)
        (1) edge [] node {} (4)
        (2) edge [] node {} (3)
        (2) edge [] node {} (4)
        (5) edge [] node {} (7)
        (5) edge [] node {} (8)
        (6) edge [] node {} (7)
        (6) edge [] node {} (8)
        (7) edge [] node {} (8)
        (1) edge [] node {} (2)
        (5) edge [] node {} (6);
  \addvmargin{1mm}
\end{ugraph}

Vervolgens bepalen we de complementgraaf:

\begin{ugraph}
  \node[state] (1)              {$1$};
  \node[state] (2) [above right of=1] {$2$};
  \node[state] (4) [right of=2] {$4$};
  \node[state] (7) [below right of=4] {$7$};
  \node[state] (3) [below of=7] {$3$};
  \node[state] (5) [below left of=3] {$5$};
  \node[state] (6) [left of=5]  {$6$};
  \node[state] (8) [above left of=6]  {$8$};
  \addvmargin{1mm}
\end{ugraph}

De complementgraaf bevat geen enkele component die uit meer dan \'e'en toestand bestaat. We concluderen dat de DFA reeds minimaal is. We zien nu ook dat de NFA minder toestanden heeft dan de equivalente minimale DFA.

\subsection{Taal is niet regulier met pompend lemma}
\label{ex:plre}

We tonen aan dat een taal $L = \{a^nb^n|n \in \nat\}$ over $\Sigma = \{a,b\}$ niet regulier is.

Stel dat er voor $L$ een pomplengte $d$ bestaat, dan beschouwen we de string $s=a^db^d$. We nemen een willekeurige opdeling van $s=xyz$ met $|y| > 0$. Er zijn drie mogelijkheden:

\begin{itemize}
\item $y$ bevat enkel $a$'s: $xyyz$ bevat dan meer $a$'s dan $b$'s, dus $xyyz \notin L$ of $xz$ bevat dan meer $b$'s dan $a$'s, dus $xz \notin L$
\item $y$ bevat enkel $b$'s: $xyyz$ bevat dan meer $b$'s dan $a$'s, dus $xyyz \notin L$ of $xz$ bevat dan meer $a$'s dan $b$'s, dus $xz \notin L$
\item $y$ is van de vorm $a^ib^j$ met $i,j \in \nat_0$: $xyyz \notin L$ want het aantal $a$'s en $b$'s blijft enkel gelijk indien $i=j$ en zelfs dan wordt de volgorde niet meer gerespecteerd, er zullen $a$'s en $b$'s door elkaar staan.
\end{itemize}

\subsection{CFG}
\label{ex:cfg}

We stellen een CFG op voor de contextvrije taal $L = \{a^nb^n|n \in \nat\}$ over $\Sigma = \{a,b\}$.

\begin{itemize}
\item $S \rightarrow aSb$
\item $S \rightarrow \epsilon$
\end{itemize}

\subsection{CNF}
\label{ex:cnf}

We stellen een CFG in CNF op voor de contextvrije taal $L = \{a^nb^n|n \in \nat\}$ over $\Sigma = \{a,b\}$.

\begin{itemize}
\item $A \rightarrow a$
\item $B \rightarrow b$
\item $C \rightarrow XB$
\item $X \rightarrow AC$
\item $X \rightarrow AB$
\item $S \rightarrow X$
\item $S \rightarrow \epsilon$
\end{itemize}

\subsection{PDA}
\label{ex:pda}

We stellen een PDA op voor de contextvrije taal $L = \{a^nb^n|n \in \nat\}$ over $\Sigma = \{a,b\}$.

\begin{pda}
  \node[initial,state]   (q1)               {$q_1$};
  \node[state]           (q2) [right of=q1] {$q_2$};
  \node[state]           (q3) [right of=q2] {$q_3$};
  \node[state,accepting] (q4) [right of=q3] {$q_4$};
  
  \path (q1) edge []           node {$\pdat{\epsilon}{\epsilon}{\$}$}       (q2)
        (q2) edge [loop above] node {$\pdat{a}{\epsilon}{a}$}               (q2)
             edge []           node {$\pdat{\epsilon}{\epsilon}{\epsilon}$} (q3)
        (q3) edge [loop above] node {$\pdat{b}{a}{\epsilon}$}               (q3)
             edge []           node {$\pdat{\epsilon}{\$}{\epsilon}$}       (q4);
  \addvmargin{1mm}
\end{pda}

Deze PDA kan niet met minder toestanden voorgesteld worden. Indien $q_3$ een eindtoestand zou zijn (en $q_2$ de starttoestand), dan zou het aantal $a$'s en $b$'s niet meer gelijk zijn, omdat de PDA dan kan stoppen voor dat het even veel $b$'s als $a$'s is tegengekomen. De strings $aaa$ of $aaab$ zouden dan onder andere geldig zijn, zo lang er minder dan of evenveel $b$'s als $a$'s zijn.

\subsection{CFG naar PDA}
\label{ex:cfgpda}

\begin{itemize}
\item $E \rightarrow a$
\item $E \rightarrow E+E$
\item $E \rightarrow E*E$
\end{itemize}

We converteren de gegeven CFG met startsymbool $E$ naar een PDA:

\begin{pda}
  \node[initial,state]   (qs)                   {$q_s$};
  \node[state]           (x)  [below=2cm of qs] {$x$};
  \node[state,accepting] (qf) [below=2cm of x]  {$q_f$};
  
  \path (qs) edge []                            node [pos=0.2] {$\pdat{\epsilon}{\epsilon}{E\$}$} (x)
        (x)  edge []                            node [pos=0.8] {$\pdat{\epsilon}{\$}{\epsilon}$}  (qf)
             edge [out=120,in=150,distance=2cm] node [left]    {$\pdat{\epsilon}{E}{a}$}          (x)
             edge [out=165,in=195,distance=2cm] node [left]    {$\pdat{\epsilon}{E}{E*E}$}        (x)
             edge [out=210,in=240,distance=2cm] node [left]    {$\pdat{\epsilon}{E}{E+E}$}        (x)
             edge [out=60,in=30,distance=2cm]   node [right]   {$\pdat{a}{a}{\epsilon}$}          (x)
             edge [out=15,in=345,distance=2cm]  node [right]   {$\pdat{*}{*}{\epsilon}$}          (x)
             edge [out=330,in=300,distance=2cm] node [right]   {$\pdat{+}{+}{\epsilon}$}          (x);
  \addvmargin{1mm}
\end{pda}

\subsection{Taal is niet contextvrij met pompend lemma}
\label{ex:plcfg}

Stel dat er voor een taal $L = \{a^nb^nc^n|n\geq\}$ over $\Sigma = \{a,b,c\}$ een pomplengte $p$ bestaat en stel dat we voor een string $s = a^pb^pc^p$ met $|s| \geq p$ een opdeling $s=uvxyz$ maken met $|vy| > 0$. Dan zijn er slechts twee mogelijkheden voor die opdeling:
\begin{itemize}
\item $v$ is van de vorm $\alpha^i$ en $y$ van de vorm $\beta^j$ waarbij $\alpha,\beta \in \Sigma$ met $k+l>0$: $uv^2xy^2z \notin L$ kan niet bestaan uit een gelijk aantal $a$'s, $b$'s en $c$'s.
\item $v$ en/of $y$ bestaan uit meer dan \'e\'en symbool uit $\Sigma$: $uv^2xy^2z \notin L$ of meer specifiek $v^2$ en/of $y^2$ bevatten symbolen die niet in de juiste volgorde staan.
\end{itemize}

We hebben aangetoond dat de taal $L$ niet contextvrij is.

\subsection{Recursieve functie in lambda calculus}
\label{ex:lambdarecurse}

Stel dat een $\lambda$-calculus niet beschikt over een optel functie, maar wel over een functie $INC$ om een getal met \'e\'en te verhogen en $DEC$ om dat getal met \'e\'en te verlagen. We stellen een recursieve functie $PLUS$ op, die een getal bij een ander getal optelt.

\begin{equation*}
  H = \lambda f,a,b.\ (IF\ (=\ b\ 0)\ a\ (f\ (INC\ a)\ (DEC\ b))) \text{ met } PLUS = H\ PLUS
\end{equation*}

$H$ is het vastpunt voor $PLUS$, nu kunnen we de defini\"erende vergelijking $PLUS = H\ PLUS$ vervangen met de vastpuntcombinator, zodat we $PLUS = Y H$ bekomen.

We kunnen nu een bewerking met deze functie uitvoeren:

\begin{equation*}
\begin{aligned}
PLUS\ 4\ 1 = \\
Y\ H\ 4\ 1 = \\
H\ (Y\ H)\ 4\ 1 = \\
\lambda f,a,b.\ (IF\ (=\ b\ 0)\ a\ (f\ (INC\ a)\ (DEC\ b)))\ (Y\ H)\ 4\ 1 \longrightarrow \\
\lambda a,b.\ (IF\ (=\ b\ 0)\ a\ ((Y\ H)\ (INC\ a)\ (DEC\ b)))\ 4\ 1 \stackrel{*}{\longrightarrow} \\
(IF\ (=\ 1\ 0)\ 4\ ((Y\ H)\ (INC\ 4)\ (DEC\ 1))) \longrightarrow \\
(IF\ FALSE\ 4\ ((Y\ H)\ (INC\ 4)\ (DEC\ 1))) \longrightarrow \\
(Y\ H)\ (INC\ 4)\ (DEC\ 1) \stackrel{*}{\longrightarrow} \\
(Y\ H)\ 5\ 0 = \\
\lambda f,a,b.\ (IF\ (=\ b\ 0)\ a\ (f\ (INC\ a)\ (DEC\ b)))\ (Y\ H)\ 5\ 0 \longrightarrow \\
\lambda a,b.\ (IF\ (=\ b\ 0)\ a\ ((Y\ H)\ (INC\ a)\ (DEC\ b)))\ 5\ 0 \stackrel{*}{\longrightarrow} \\
(IF\ (=\ 0\ 0)\ 5\ ((Y\ H)\ (INC\ 5)\ (DEC\ 0))) \longrightarrow \\
(IF\ TRUE\ 5\ ((Y\ H)\ (INC\ 5)\ (DEC\ 0))) \longrightarrow \\
5
\end{aligned}
\end{equation*}