\section{Examenvragen}

\subsection{Reguliere talen}

\subsubsection{Vraag 1 (Bijvraag)}

\begin{center}
\renewcommand{\arraystretch}{1.5}
\begin{tabular}{>{\centering\arraybackslash}m{5cm}>{\centering\arraybackslash}m{1cm} >{\centering\arraybackslash}m{5cm}}
\begin{nfa}
  \node[state] (A)                     {$A$};
  \node[state] (X)  [below right of=A] {$X$};
  \node[state] (B)  [above right of=X] {$B$};
  
  \path (A) edge [bend left]  node {$E_4$} (B)
            edge [bend right] node {$E_1$} (X)
        (X) edge [loop below] node {$E_2$} (E)
            edge [bend right] node {$E_3$} (B);
  \addvmargin{1mm}
\end{nfa} & $\longrightarrow$ & \begin{nfa}
  \node[state] (A)                   {$A$};
  \node[state] (B)  [right=3cm of A] {$B$};
  
  \path (A) edge []  node {$E_4|E_1(E_2)^*E_3$} (B);
  \addvmargin{1mm}
\end{nfa}
\end{tabular}
\end{center}

\textit{Stel dat het linkse deel toestand B niet bevat, de boog met $E_3$ van $X$ naar $A$ gaat en de boog met $E_4$ van $A$ naar zichzelf gaat. Kan dit omgevormd worden op methode gelijkaardig aan die beschreven is in de cursus?}

\begin{center}
\begin{nfa}
  \node[state] (A)              {$A$};
  \node[state] (X) [right of=A] {$X$};
  
  \path (A) edge [bend left]  node {$E_1$} (X)
            edge [loop left]  node {$E_4$} (A)
        (X) edge [bend left]  node {$E_3$} (A)
            edge [loop right] node {$E_2$} (X);
  \addvmargin{1mm}
\end{nfa}
\end{center}

Ja, we elimineren eerst het pad van $X$ naar $A$. We voeren een nieuwe toestand $Y$ in en gebruiken de standaard eleminatie om het pad van $Y$ naar $X$ door $A$ te behandelen:

\begin{center}
\renewcommand{\arraystretch}{1.5}
\begin{tabular}{>{\centering\arraybackslash}m{5cm}>{\centering\arraybackslash}m{1cm} >{\centering\arraybackslash}m{5cm}}
\begin{nfa}
  \node[state] (A)                   {$A$};
  \node[state] (X) [above right=.25cm and 2cm of A] {$X$};
  \node[state] (Y) [below right=.25cm and 2cm of A] {$Y$};
  
  \path (A) edge [bend left]  node {$E_1$}      (X)
            edge [loop left]  node {$E_4$}      (A)
        (X) edge [bend left]  node {$\epsilon$} (Y)
            edge [loop right] node {$E_2$}      (X)
        (Y) edge [bend left]  node {$E_3$}      (A)
            edge [bend left]  node {$\epsilon$} (X);
  \addvmargin{1mm}
\end{nfa} & $\longrightarrow$ & \begin{nfa}
  \node[state] (X) []                {$X$};
  \node[state] (Y) [below=.75cm of X] {$Y$};
  
  \path (X) edge [bend left]  node {$\epsilon$}               (Y)
            edge [loop right] node {$E_2$}                    (X)
        (Y) edge [bend left]  node {$\epsilon|E_3(E_4)^*E_1$} (X);
  \addvmargin{1mm}
\end{nfa}
\end{tabular}
\end{center}

Nu we de boog van $X$ naar $A$ ge\"elemineerd hebben, kunnen we de resterende bewerkingen uitvoeren om de lussen van de knopen op de boog tussen $A$ en $X$ te plaatsen:

\begin{center}
\renewcommand{\arraystretch}{1.5}
\begin{tabular}{>{\centering\arraybackslash}m{5cm}>{\centering\arraybackslash}m{1cm} >{\centering\arraybackslash}m{5cm}}
\begin{nfa}
  \node[state] (A)              {$A$};
  \node[state] (X) [right of=A] {$X$};
  
  \path (A) edge []           node {$E_1$} (X)
            edge [loop left]  node {$E_4$} (A)
        (X) edge [loop above] node {$E_2|E_3(E_4)^*E_1$} (X);
  \addvmargin{1mm}
\end{nfa} & $\longrightarrow$ & \begin{nfa}
  \node[state] (A)              {$A$};
  \node[state] (X) [right=4cm of A] {$X$};
  
  \path (A) edge [] node {$(E_4)^*E_1(E_2|E_3(E_4)^*E_1)^*$} (X);
  \addvmargin{1mm}
\end{nfa}
\end{tabular}
\end{center}

\subsubsection{Vraag 2}

\textit{Geef de definitie van een Myhil-Nerode relatie over een taal $L$, of zoals we noteren een $MN(L)$-relatie.}

Een Myhill-Nerode relatie voor een taal $L$ ($MN(L)$-relatie) is een equivalentierelatie $\sim_X$ die voldoet aan de volgende eigenschappen:
  \begin{itemize}
  \item $\sim_X$ is rechts congruent\\$\forall x, y \in \Sigma^*, a \in \Sigma: x \sim_X y \Rightarrow xa \sim_X ya$
  \item $\sim_X$ verfijnt $\sim_L$\\$x \sim_X y \Rightarrow x \sim_L y$
  \item $\sim_X$ heeft een eindige index\\Het aantal equivalentieklassen van $\sim_X$ is eindig.
  \end{itemize}

\textit{Bewijs vervolgens dat een $MN(L)$-relatie bestaat als en slechts als $L$ regulier is.}

  Het bewijs wordt geleverd door aan te tonen dat het mogelijk is om een $MN(L)$-relatie te construeren uit een DFA en een DFA uit een $MN(L)$-relatie.
  \begin{itemize}
  \item Voor een gegeven DFA $D$ kan men een equivalentierelatie $\sim_D$ construeren waarvoor geldt
  \begin{equation*}
  \forall x, y \in \Sigma^*: x \sim_D y \Leftrightarrow \delta^*(q_s, x) = \delta^*(q_s, y)
  \end{equation*}
  \begin{itemize}
  \item $\sim_D$ is rechts congruent omdat geldt
  \begin{equation*}
  \forall x, y \in \Sigma^*, a \in \Sigma: xa \sim_D ya \Leftrightarrow \delta(\delta^*(q_s, x), a) = \delta(\delta^*(q_s, y), a)
  \end{equation*}
  \item $\sim_D$ verfijnt $\sim_{L_D}$, omdat $\sim_D$ een equivalentieklasse genereert voor alle strings die aanvaard worden door $D$ per eindtoestand en een klasse voor elke string die eindigt in een andere toestand. Daardoor worden de verzamelingen $L$ en $\overline{L}$ opgedeeld in kleinere disjuncte verzamelingen. De verzameling van die equivalentieklassen is een partitie van $\Sigma^*$ die $\sim_{L_D}$ verfijnt.
  \item $\sim_D$ heeft een eindige index omdat het aantal toestanden in $D$ eindig is.
  \end{itemize}
  \item Voor een $MN(L)$-relatie kan men een DFA ($Q$, $\Sigma$, $\delta$, $q_s$, $F$) construeren die $L$ bepaalt, waarvoor geldt
  \begin{itemize}
  \item $Q = \{x_\sim|x \in \Sigma^*\}$\\
  De toestanden van de DFA zijn gelijk aan de equivalentieklassen van de $MN(L)$-relatie, dat aantal equivalentieklassen is eindig.
  \item $\delta(x_\sim, a) = (xa)_\sim$\\
  Volgt uit de rechtse congruentie eigenschap van de equivalentierelatie.
  \item $q_s = \epsilon_\sim$
  Een starttoestand wordt bereikt met $\epsilon$.
  \item $F = \{x_\sim|x \in L\}$\\
  Een eindtoestand wordt bereikt door elke string in $L$. Het aantal eindtoestanden is eindig omdat $F \subseteq Q$.
  \end{itemize}
  Tenslotte bewijzen we met inductie voor een willekeurige string $x \in \Sigma^*$ dat geldt
  \begin{equation*}
  \forall x \in \Sigma^*: x \in L_{DFA} \triangleq \delta^*(\epsilon_\sim,x) \in F \Leftrightarrow x \in L \triangleq x_\sim \in F
  \end{equation*}
  \begin{itemize}
  \item Voor een string $x$ met een lengte $l = 0$:
  \begin{equation*}
  \delta^*(\epsilon_\sim,\epsilon) = \delta(\epsilon_\sim,\epsilon) = \epsilon_\sim
  \end{equation*}
  \item Stel dat de bewering geldt voor een string met willekeurige lengte $l$, dan geldt ze ook voor $xa$ met $a \in \Sigma$:
  \begin{equation*}
  \delta^*(\epsilon_\sim,xa) = \delta(x_\sim,a) = (xa)_\sim
  \end{equation*}
  \end{itemize}
  \end{itemize}

\textit{Bestaat er voor een taal $L$ soms meer dan één $MN(L)$-relatie?}

Een taal $L$ waarvoor een $MN(L)$-relatie bestaat moet regulier zijn. Een mogelijke $MN(L)$-relatie voor een reguliere taal is $\sim_{DFA}$. Die equivalentierelatie wordt bepaald door alle strings die vanuit de begintoestand van de $DFA$ een willekeurige toestand bereiken te groeperen per toestand.
\begin{equation*}
\forall x, y \in \Sigma^*: x \sim_{DFA} y \Leftrightarrow \delta^*(q, x) = \delta^*(q, y)
\end{equation*}
We kunnen voor de taal $L$ een equivalente DFA opstellen die dezelfde taal bepaalt, maar niet isomorf is. De equivalentierelatie $\sim_{DFA}$ voor die DFA zal $\Sigma^*$ anders partitioneren. Bijgevolg bestaat er meer dan \'e\'en $MN(L)$-relatie voor een taal $L$.

\subsection{Contextvrije talen}

\subsubsection{Vraag 1 (Bijvraag)}

\textit{Volgens een constructie in de cursus kan een PDA bij een overgang meerdere elementen per keer pushen, waarom mag dit?}

% TODO: Zie constructie PDA uit CFG