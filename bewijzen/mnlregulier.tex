  Het bewijs wordt geleverd door aan te tonen dat het mogelijk is om een $MN(L)$-relatie te construeren uit een DFA en een DFA uit een $MN(L)$-relatie.
  \begin{itemize}
  \item Voor een gegeven DFA $D$ kan men een equivalentierelatie $\sim_D$ construeren waarvoor geldt
  \begin{equation*}
  \forall x, y \in \Sigma^*: x \sim_D y \Leftrightarrow \delta^*(q_s, x) = \delta^*(q_s, y)
  \end{equation*}
  \begin{itemize}
  \item $\sim_D$ is rechts congruent omdat geldt
  \begin{equation*}
  \forall x, y \in \Sigma^*, a \in \Sigma: xa \sim_D ya \Leftrightarrow \delta(\delta^*(q_s, x), a) = \delta(\delta^*(q_s, y), a)
  \end{equation*}
  \item $\sim_D$ verfijnt $\sim_{L_D}$, omdat $\sim_D$ een equivalentieklasse genereert voor alle strings die aanvaard worden door $D$ per eindtoestand en een klasse voor elke string die eindigt in een andere toestand. Daardoor worden de verzamelingen $L$ en $\overline{L}$ opgedeeld in kleinere disjuncte verzamelingen. De verzameling van die equivalentieklassen is een partitie van $\Sigma^*$ die $\sim_{L_D}$ verfijnt.
  \item $\sim_D$ heeft een eindige index omdat het aantal toestanden in $D$ eindig is.
  \end{itemize}
  \item Voor een $MN(L)$-relatie kan men een DFA ($Q$, $\Sigma$, $\delta$, $q_s$, $F$) construeren die $L$ bepaalt, waarvoor geldt
  \begin{itemize}
  \item $Q = \{x_\sim|x \in \Sigma^*\}$\\
  De toestanden van de DFA zijn gelijk aan de equivalentieklassen van de $MN(L)$-relatie, dat aantal equivalentieklassen is eindig.
  \item $\delta(x_\sim, a) = (xa)_\sim$\\
  Volgt uit de rechtse congruentie eigenschap van de equivalentierelatie.
  \item $q_s = \epsilon_\sim$
  Een starttoestand wordt bereikt met $\epsilon$.
  \item $F = \{x_\sim|x \in L\}$\\
  Een eindtoestand wordt bereikt door elke string in $L$. Het aantal eindtoestanden is eindig omdat $F \subseteq Q$.
  \end{itemize}
  Tenslotte bewijzen we met inductie voor een willekeurige string $x \in \Sigma^*$ dat geldt
  \begin{equation*}
  \forall x \in \Sigma^*: x \in L_{DFA} \triangleq \delta^*(\epsilon_\sim,x) \in F \Leftrightarrow x \in L \triangleq x_\sim \in F
  \end{equation*}
  \begin{itemize}
  \item Voor een string $x$ met een lengte $l = 0$:
  \begin{equation*}
  \delta^*(\epsilon_\sim,\epsilon) = \delta(\epsilon_\sim,\epsilon) = \epsilon_\sim
  \end{equation*}
  \item Stel dat de bewering geldt voor een string met willekeurige lengte $l$, dan geldt ze ook voor $xa$ met $a \in \Sigma$:
  \begin{equation*}
  \delta^*(\epsilon_\sim,xa) = \delta(x_\sim,a) = (xa)_\sim
  \end{equation*}
  \end{itemize}
  \end{itemize}