Stel dat een $\lambda$-calculus niet beschikt over een optel functie, maar wel over een functie $INC$ om een getal met \'e\'en te verhogen en $DEC$ om dat getal met \'e\'en te verlagen. We stellen een recursieve functie $PLUS$ op, die een getal bij een ander getal optelt.

\begin{equation*}
  H = \lambda f,a,b.\ (IF\ (=\ b\ 0)\ a\ (f\ (INC\ a)\ (DEC\ b))) \text{ met } PLUS = H\ PLUS
\end{equation*}

$H$ is het vastpunt voor $PLUS$, nu kunnen we de defini\"erende vergelijking $PLUS = H\ PLUS$ vervangen met de vastpuntcombinator, zodat we $PLUS = Y H$ bekomen.

We kunnen nu een bewerking met deze functie uitvoeren:

\begin{equation*}
\begin{aligned}
PLUS\ 4\ 1 = \\
Y\ H\ 4\ 1 = \\
H\ (Y\ H)\ 4\ 1 = \\
\lambda f,a,b.\ (IF\ (=\ b\ 0)\ a\ (f\ (INC\ a)\ (DEC\ b)))\ (Y\ H)\ 4\ 1 \longrightarrow \\
\lambda a,b.\ (IF\ (=\ b\ 0)\ a\ ((Y\ H)\ (INC\ a)\ (DEC\ b)))\ 4\ 1 \stackrel{*}{\longrightarrow} \\
(IF\ (=\ 1\ 0)\ 4\ ((Y\ H)\ (INC\ 4)\ (DEC\ 1))) \longrightarrow \\
(IF\ FALSE\ 4\ ((Y\ H)\ (INC\ 4)\ (DEC\ 1))) \longrightarrow \\
(Y\ H)\ (INC\ 4)\ (DEC\ 1) \stackrel{*}{\longrightarrow} \\
(Y\ H)\ 5\ 0 = \\
\lambda f,a,b.\ (IF\ (=\ b\ 0)\ a\ (f\ (INC\ a)\ (DEC\ b)))\ (Y\ H)\ 5\ 0 \longrightarrow \\
\lambda a,b.\ (IF\ (=\ b\ 0)\ a\ ((Y\ H)\ (INC\ a)\ (DEC\ b)))\ 5\ 0 \stackrel{*}{\longrightarrow} \\
(IF\ (=\ 0\ 0)\ 5\ ((Y\ H)\ (INC\ 5)\ (DEC\ 0))) \longrightarrow \\
(IF\ TRUE\ 5\ ((Y\ H)\ (INC\ 5)\ (DEC\ 0))) \longrightarrow \\
5
\end{aligned}
\end{equation*}