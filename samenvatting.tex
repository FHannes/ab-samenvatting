\documentclass[a4paper]{article}

\usepackage[dutch]{babel}
\usepackage[pdftex]{graphicx}
\usepackage{amsfonts}
\usepackage{amsmath}
\usepackage{amsthm}
\usepackage{xcolor}
\usepackage{mdframed}

\definecolor{silver}{rgb}{0.95,0.95,0.95}

\newtheorem{tdefinitie}{Definitie}[section]
\newenvironment{definitie}[1]%
  {\begin{mdframed}[backgroundcolor=silver,
    topline=false,
    rightline=false,
    leftline=false,
    bottomline=false]\begin{tdefinitie}#1\\\normalfont}%
  {\end{tdefinitie}\end{mdframed}}
  
\newcommand{\perm}{\ensuremath{\mathcal{P}}}
\newcommand{\nat}{\ensuremath{\mathbb{N}}}
\newcommand{\of}{\ensuremath{\text{ of }}}
\newcommand{\en}{\ensuremath{\text{ en }}}

\begin{document}

\begin{titlepage}
    \newpage
    \thispagestyle{empty}
    \frenchspacing
    \hspace{-0.2cm}
    \includegraphics[height=3.4cm]{assets/sedes}
    \hspace{0.2cm}
    \rule{0.5pt}{3.4cm}
    \hspace{0.2cm}
    \begin{minipage}[b]{8cm}
        \large{Katholieke\newline Universiteit\newline Leuven}\smallskip\newline
        \large{}\smallskip\newline
        \textbf{Department \newline Computerwetenschappen}\smallskip
    \end{minipage}
    \hspace{\stretch{1}}
    \vspace*{3.2cm}\vfill
    \begin{center}
        \begin{minipage}[t]{\textwidth}
            \begin{center}
                \large{\rm{\textbf{\uppercase{Samenvatting}}}}\\
                \large{\rm{Automaten en Berekenbaarheid [G0P84a]}}
            \end{center}
        \end{minipage}
    \end{center}
    \vfill
    \hfill\makebox[3cm][l]{%
        \vbox to 7cm{\vfill\noindent
            {\rm \textbf{Fr\'ed\'eric Hannes}}\\[2mm]
            {\rm Academisch jaar 2015--2016}
        }
    }
\end{titlepage}

\tableofcontents

\newpage

\section{Voorkennis}

\subsection{Griekse letters}

\begin{tabular}{l|l|l}
	Alpha & $\alpha$ & \\
	Beta & $\beta$ & \\
	Delta & $\delta$ & $\Delta$ \\
	Epsilon & $\varepsilon, \epsilon$ & \\
	Gamma & $\gamma$ & $\Gamma$ \\
	Mu & $\mu$ & \\
	Omega & $\omega$ & $\Omega$ \\
	Phi & $\varphi, \phi$ & $\Phi$ \\
	Rho & $\rho$ & \\
	Sigma & $\sigma$ & $\Sigma$ \\
	Tau & $\tau$ & \\
	Theta & $\theta$ & $\Theta$ \\
	Xi & $\xi$ & $\Xi$ \\
\end{tabular}

\subsection{Verzamelingenleer}

\begin{definitie}{Een algebra van een verzameling S}
Een algebra is een verzameling met daarop een aantal inwendige operaties.
\end{definitie}

\begin{definitie}{De unie van twee verzamelingen}
  De unie van twee verzamelingen $S_1$ en $S_2$ is de verzameling die alle elementen bevat die in $S_1$ en/of $S_2$ zitten. \\
  \\ $S_1 \cup S_2 = \{x|x \in S_1 \of x \in S_2\}$
\end{definitie}

\begin{definitie}{De doorsnede van twee verzamelingen}
  De doorsnede van twee verzamelingen $S_1$ en $S_2$ is de verzameling die alle elementen bevat die in zowel $S_1$ als $S_2$ zitten. \\
  \\ $S_1 \cup S_2 = \{x|x \in S_1 \en x \in S_2\}$
\end{definitie}

\begin{definitie}{Het complement van een verzameling}
  Het complement van een verzameling $S$ van alle elementen uit het domein die niet in $S$ zitten. \\
  \\ $\overline{S} = \{x|x \notin S\}$
\end{definitie}

\begin{definitie}{Een machtsverzameling $\perm(S)$ van een verzameling S}
  Een machtsverzameling $\perm(S)$ is een verzameling van alle deelverzamelingen die gevormd kunnen worden met de elementen in S, inclusief de lege verzameling en de verzameling A.\\ \\ $\perm(S) = \{A|A \subseteq S\}$
\end{definitie}

\section{Talen}

\begin{definitie}{Een alfabet $\Sigma$}
  Een verzameling van tekens.
\end{definitie}

\begin{definitie}{Een string $s$ over een alfabet $\Sigma$}
  Een opeenvolging nul, \'e\'en of meerdere elementen uit $\Sigma$. \\
  \\ $s = a_1a_2a_3...a_n$ met $a_i \in \Sigma, n \in \nat$
  \\ $s \in L$
\end{definitie}

\begin{definitie}{De lege string $\epsilon$ over een alfabet $\Sigma$}
  De lege string is een string die bestaat uit nul elementen.
\end{definitie}

\begin{definitie}{Een taal $L$ over een alfabet $\Sigma$}
  Een verzameling van strings over $\Sigma$.\\
  \\ $L \subset \Sigma^*$
  \\ $L \in \perm(\Sigma^*)$
\end{definitie}

\begin{definitie}{Concatenatie van twee talen}
  $L_1L_2$ is de concatenatie van twee talen $L_1$ en $L_2$ over hetzelfde alfabet $\Sigma$. \\
  \\ $L_1L_2 = \{xy|x \in L_1, y \in L_2\}$
  \\ $L^n = \{x_1x_2x_3...x_n|x_i \in L, n \in \nat\}$
  \\ $L^0 = \{\epsilon\}$
\end{definitie}

\begin{definitie}{De Kleene ster $L^*$ van een taal $L$}
  $L^* = \bigcup^\infty_{n=0}L^n$
  \\ $L^+ = LL^*$
\end{definitie}

\begin{definitie}{De verzameling $\Sigma^*$ van alle strings over $\Sigma$}
  $\Sigma^*$ is een verzameling met alle strings die gevormd kunnen worden uit het alfabet $\Sigma$. Elke taal over $\Sigma$ is een deelverzameling van $\Sigma^*$. \\
  \\ $\Sigma^* = \{a_1a_2a_3...a_n|a_i \in \Sigma, n \in \nat\}$
\end{definitie}

\begin{definitie}{De verzameling $\perm(\Sigma^*)$ van alle talen over $\Sigma$}
  $\perm(\Sigma^*)$ is een verzameling met alle talen die gevormd kunnen worden uit het alfabet $\Sigma$. Elke taal over $\Sigma$ is een element van $\perm(\Sigma^*)$.
\end{definitie}

De verzameling $\perm(\Sigma^*)$ wordt een algebra door de definitie van verschillende inwendige operaties, waaronder:
\begin{itemize}
\item Unie: $L_1 \cup L_2 \in \perm(\Sigma^*)$ met $L_1 \in \perm(\Sigma^*), L_2 \in \perm(\Sigma^*)$
\item Doorsnede: $L_1 \cap L_2 \in \perm(\Sigma^*)$ met $L_1 \in \perm(\Sigma^*), L_2 \in \perm(\Sigma^*)$
\item Complement: $\overline{L} \in \perm(\Sigma^*)$ met $L \in \perm(\Sigma^*)$
\end{itemize}

\section{Reguliere talen}
  
\end{document}
