Een enumeratormachine is een variant van een Turingmachine (definitie \ref{def:tm}) met een extra enumeratortoestand $q_e$ en een overgangsfunctie met als signatuur \bm{$\delta: Q \times \Gamma \rightarrow Q \times \Gamma \times \Gamma_\epsilon \times \{L, R, S\}$}. Bij een overgang wordt voor $\Gamma_\epsilon$ in de signatuur een teken op een outputband geschreven. Bij het bereiken van $q_e$ wordt een outputmarker op de outputband geschreven die een outputstring afscheidt van de volgende.